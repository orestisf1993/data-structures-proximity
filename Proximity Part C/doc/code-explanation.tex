\chapter{Υλοποίηση Των Συναρτήσεων}
\section{Κλάση \texttt{MinMaxPlayer}}
\subsection{Συνάρτηση \texttt{findEmptyTiles()}}\label{fun:findEmptyTiles}
\begin{lstlisting}[title={Declaration της συνάρτησης}]
private static ArrayList<Tile> findEmptyTiles(final Board board, final int nextDepth)
\end{lstlisting}
Δέχεται το \lstinline!board! και το βάθος \lstinline!nextDepth! του υποδέντρου που δημιουργείται και επιστρέφει ένα
\lstinline!ArrayList<Tile>! με τα ελεύθερα πλακίδια στο \lstinline!board! απορρίπτοντας τα πολλαπλά "μοναχικά" πλακίδια.

Η δομή της συνάρτησης είναι η εξής:
\begin{itemize}
\item Δημιουργούμε μία λίστα για να φορτώνουμε τα πλακίδια που θα επιστρέψουμε.
\begin{lstlisting}[style=chunk]
final int loneTilesLimit = MAX_DEPTH + 1 - nextDepth;
\end{lstlisting}

\item Πραγματοποιούμε μια επανάληψη πάνω σε κάθε πλακίδιο του \lstinline!board!.
\begin{lstlisting}[style=chunk]
for (int i = 0; i < ProximityUtilities.NUMBER_OF_COLUMNS; i++) {
    for (int j = 0; j < ProximityUtilities.NUMBER_OF_ROWS; j++) {
        final Tile tile = board.getTile(i, j);
\end{lstlisting}

\item Για να συμπεριλάβουμε ένα \lstinline!tile! θα πρέπει να είναι κενό
(\lstinline!tile.getPlayerId() == 0!)
και να μην έχουμε ήδη συμπεριλάβει αρκετά "μοναχικά" πλακίδια.
Ένα πλακίδιο το θεωρούμε "μοναχικό" αν όλοι οι γείτονες του είναι κενοί.
Αυτό γίνεται με την συνάρτηση \lstinline!tileIsLone()!.
Στο συνολικό δέντρο θέλουμε να έχουμε \lstinline!MAX_DEPTH! "μοναχικά" πλακίδια έτσι το όριο μέσα στη συνάρτηση είναι:
\lstinline!final int loneTilesLimit = MAX_DEPTH + 1 - nextDepth;!
Με αυτή την επιλογή μειώνουμε σημαντικά τον αριθμό των πλακιδίων για τον οποίο δημιουργούμε το δέντρο μας,
βελτιώνοντας σημαντικά τον χρόνο απόφασης για κάθε κίνηση.
Η ποιότητα της κίνηση δεν μειώνεται γιατί το evaluation μεταξύ των μοναχικών πλακιδίων είναι γενικά ισάξιο.

\item Επιστρέφουμε την λίστα.
\begin{lstlisting}[style=chunk]
return emptyTiles;
\end{lstlisting}
\end{itemize}

\subsection{Συνάρτηση \texttt{createSubTree()}}
\begin{lstlisting}[style=declaration]
void createSubTree(final Node77968125 parent)
\end{lstlisting}
Η συνάρτηση αυτή δημιουργεί ένα υποδέντρο με συνολικό βάθος \lstinline!MAX_DEPTH!.
Δέχεται ως όρισμα τον κόμβο \lstinline!parent! πού είναι ο κόμβος κάτω από τον οποίο αρχίζει το δέντρο.

Η δημιουργία του υποδέντρου γίνεται ως εξής:
\begin{itemize}
\item Βρίσκουμε τα ελεύθερα πλακίδια μέσω της \hyperref[fun:findEmptyTiles]{\lstinline!findEmptyTiles()!}.
Αυτά τα πλακίδια θα γίνουν τα παιδιά του \lstinline!parent!.
\begin{lstlisting}[style=chunk]
final Board board = parent.getNodeBoard();
final int nextDepth = parent.getNodeDepth() + 1;
final int nodeId = nodeLevelIsOurs(nextDepth) ? id : opponentId;
final int s = nextNumbersToBePlayed[nextDepth - 1];
final ArrayList<Tile> emptyTiles = findEmptyTiles(board, nextDepth);
\end{lstlisting}

\item Αρχίζουμε μία επανάληψη για κάθε πλακίδιο στη λίστα
\begin{lstlisting}[style=chunk]
for (final Tile emptyTile : emptyTiles){
\end{lstlisting}

\item Βρίσκουμε το επόμενο \lstinline!board! μέσω της \lstinline!ProximityUtilities.boardAfterMove()!.
\begin{lstlisting}[style=chunk]
final int x = emptyTile.getX();
final int y = emptyTile.getY();
final int[] move = new int[] { x, y };

// Simulate putting a tile on this spot on the parent node board.
final Board nextBoard = ProximityUtilities.boardAfterMove(nodeId, board, x, y, s);
\end{lstlisting}

\item Δημιουργούμε ένα νέο κόμβο στο δέντρο και τον περνάμε σαν παιδί του \lstinline!parent!.
\begin{lstlisting}[style=chunk]
// Create the new node.
final Node77968125 newNode = new Node77968125(parent, nextDepth, move, nextBoard);

// Add the node as child of the parent node.
parent.addChild(newNode);
\end{lstlisting}

\item Άμα δεν έχουμε ξεπεράσει το \lstinline!MAX_DEPTH! καλούμε την \lstinline!createSubTree()! αναδρομικά με νέο πατέρα-κόμβο τον κόμβο που μόλις δημιουργήσαμε.
\begin{lstlisting}[style=chunk]
// Add opponent's branches.
if (nextDepth < MAX_DEPTH) {
    createSubTree(newNode);
}
\end{lstlisting}
\end{itemize}

\subsection{Συνάρτηση \texttt{chooseMinMaxMove()}}
\begin{lstlisting}[style=declaration]
int[] chooseMinMaxMove(final Node77968125 node)
\end{lstlisting}
Για τη διάσχιση του δέντρου ακολουθήσαμε τον αλγόριθμο depth first.
Η διάσχιση ξεκινά από τη ρίζα και εξερευνά όσο το δυνατόν περισσότερο κατά μήκος κάθε κλαδί του δέντρου μέχρι να φτάσει σε αδιέξοδο.
Η δομή της συνάρτησης είναι η εξής:
\begin{itemize}
\item Καλούμε την \lstinline!getChildren()! για το αντικείμενο \lstinline!node! της κλάσης \lstinline!Node77968125! και παίρνουμε ένα
\lstinline!ArrayList! με τα παιδιά του \lstinline!node!.

\item Αν ο \lstinline!node! δεν έχει παιδιά (τερματικός κόμβος) καλούμε την
συνάρτηση \lstinline!evaluation()! για το \lstinline!node! και επιστρέφουμε τις συντεταγμένες του \lstinline!node! μέσω της \lstinline!getNodeMove()!.
\begin{lstlisting}[style=chunk]
final ArrayList<Node77968125> children = node.getChildren();
if (children.isEmpty()) {
    node.evaluate();
    return node.getNodeMove();
}
\end{lstlisting}

\item Καλούμε τη \lstinline!nodeLevelIsOurs()! η οποία επιστρέφει \lstinline!true! αν στο επίπεδο που βρίσκεται το
\lstinline!node! παίζουμε εμείς και αποθηκεύουμε το αποτέλεσμα στη
\lstinline!minimizingPlayer!.
\begin{lstlisting}[style=chunk]
// if node is ours, it's child will not be.
final boolean minimizingPlayer = nodeLevelIsOurs(node);
\end{lstlisting}

\item Δηλώνουμε το αντικείμενο \lstinline!bestNode! της κλάσης
\lstinline!Node77968125! το οποίο αρχικοποιείται με το πρώτο παιδί του \lstinline!node!.
\begin{lstlisting}[style=chunk]
Node77968125 bestNode = children.get(0);
\end{lstlisting}

\item \sloppy Δημιουργούμε
τη μεταβλητή \lstinline!bestValue! η οποία παίρνει τιμή \lstinline!Double.POSITIVE_INFINITY!
αν η μεταβλητή \lstinline!minimizingPlayer! είναι \lstinline!true!,
διαφορετικά \lstinline!Double.NEGATIVE_INFINITY!.
\begin{lstlisting}[style=chunk]
double bestValue = minimizingPlayer ? Double.POSITIVE_INFINITY : Double.NEGATIVE_INFINITY;
\end{lstlisting}

\item Ξεκινάει μία λούπα στην οποία
παίρνουμε με τη σειρά όλα τα παιδιά του \lstinline!node!.
\begin{lstlisting}[style=chunk]
for (final Node77968125 child : children) {
\end{lstlisting}

\item Καλούμε αναδρομικά την
\lstinline!chooseMinMaxMove()! για το \lstinline!child!
και παίρνουμε το νέο \lstinline!evaluation! του κόμβου.
\begin{lstlisting}[style=chunk]
chooseMinMaxMove(child);
final double evaluation = child.getNodeEvaluation();
\end{lstlisting}

\item Δημιουργούμε την \lstinline!boolean! μεταβλητή \lstinline!isBestValue! η οποία
παίρνει την τιμή της:
\begin{itemize}
\item Από την ανίσωση \lstinline!evaluation < bestValue! αν η \lstinline!minimizingPlayer! είναι \lstinline!true!.
\item Από την ανίσωση \lstinline!evaluation > bestValue! αν η \lstinline!minimizingPlayer! είναι \lstinline!false!.
\end{itemize}
Αν είμαστε σε κόμβο που θέλουμε
μεγιστοποίηση διαλέγουμε το node που μας μεγιστοποιεί το \lstinline!evaluation!
αντίθετα επιλέγουμε τον κόμβο με το μικρότερο \lstinline!evaluation!.

\label{item:bestValue}
Αν η συνθήκη είναι \lstinline!true! αποθηκεύουμε στο αντικείμενο \lstinline!bestNode! το \lstinline!child! και στο \lstinline!bestValue! την \lstinline!evaluation!.
\begin{lstlisting}[style=chunk]
final boolean isBestValue = minimizingPlayer ? evaluation < bestValue
        : evaluation > bestValue;
if (isBestValue) {
    bestNode = child;
    bestValue = evaluation;
}
\end{lstlisting}

\item Βγαίνουμε από την λούπα και ανανεώνουμε την αξιολόγηση του \lstinline!node! σύμφωνα με την τιμή που βρήκαμε
\hyperref[item:bestValue]{προηγουμένως}.
\begin{lstlisting}[style=chunk]
}
node.setNodeEvaluation(bestValue);
return bestNode.getNodeMove();
\end{lstlisting}
\end{itemize}

\subsection{Συνάρτηση \texttt{getNextMove()}}
\begin{lstlisting}[style=declaration]
public int[] getNextMove(final Board board, final int randomNumber)
\end{lstlisting}
Η συνάρτηση αυτή δέχεται ως ορίσματα ένα αντικείμενο τύπου \lstinline!Board! και
την μεταβλητή \lstinline!randomNumber! και επιστρέφει έναν
μονοδιάστατο πίνακα με τις συντεταγμένες \lstinline!x,y!
που αντιστοιχούν στην θέση με την καλύτερη αξιολόγηση.

Η συνάρτηση κάνει τα εξής:
\begin{itemize}
\item Αποθηκεύει τον πίνακα με τις επόμενες κινήσεις ώστε να χρησιμοποιηθεί από τις υπόλοιπες συναρτήσεις.
\begin{lstlisting}[style=chunk]
nextNumbersToBePlayed = Board.getNextTenNumbersToBePlayed();
\end{lstlisting}

\item Δημιουργεί τον root κόμβο.
\begin{lstlisting}[style=chunk]
final Node77968125 root = new Node77968125(board, id);
\end{lstlisting}

\item Δημιουργεί το δέντρο κάτω από τον root κόμβο.
\begin{lstlisting}[style=chunk]
createSubTree(root);
\end{lstlisting}

\item Καλεί τον MinMax στο δέντρο και επιστρέφει το αποτέλεσμα.
\begin{lstlisting}[style=chunk]
return chooseMinMaxMove(root);
\end{lstlisting}
\end{itemize}

\section{Κλάση \texttt{Node77968125}}
\subsection{Συνάρτηση \texttt{evaluate()}}
\begin{lstlisting}[style=declaration]
void evaluate()
\end{lstlisting}
Πρόκειται για την συνάρτηση που αξιολογεί το ταμπλό ενός κόμβου, \lstinline!nodeBoard!.

Η συνάρτηση αξιολόγησης άλλαξε αρκετά σε σχέση με την προηγούμενη εργασία.
Η παλιά συνάρτηση αξιολογούσε μια κίνηση, δηλαδή την αλλαγή στο ταμπλό.
Έτσι, σε ένα δέντρο με βάθος 2 και πάνω αξιολογώντας τους τερματικούς κόμβους δεν θα περιλαμβάναμε τις αλλαγές που έγιναν στους κόμβους βάθους 1.
Για αυτό το λόγο τροποποιήσαμε την συνάρτηση ώστε να αξιολογείται ολόκληρο το ταμπλό στην τρέχουσα κατάσταση.

Η δομή της συνάρτησης είναι ως εξής:
\begin{itemize}
\item Η αξιολόγηση γίνεται πάνω σε κάθε μη-κενό πλακίδιο του ταμπλό.
\begin{lstlisting}[style=chunk]
for (int i = 0; i < ProximityUtilities.NUMBER_OF_COLUMNS; i++) {
    for (int j = 0; j < ProximityUtilities.NUMBER_OF_ROWS; j++) {
        final Tile tile = nodeBoard.getTile(i, j);
        final int tileId = tile.getPlayerId();
        if (tileId == 0) {
            continue;
        }
\end{lstlisting}

\item \label{item:nextEnemies} \sloppy
Υπολογίζουμε για κάθε πλακίδιο τους επόμενους εχθρούς του από τις κινήσεις που θα παιχτούν.
Στον πίνακα \lstinline!nextTen! που παίρνουμε μέσω της \lstinline!Board.getNextTenNumbersToBePlayed()!
τα στοιχεία \lstinline!0,2,4,...! είναι δικά μας ενώ τα \lstinline!1,3,5,...! είναι του αντιπάλου.
Σε έναν κόμβο βάθους \lstinline!nodeDepth!:
\begin{itemize}
\item Το πλακίδιο που τοποθετεί ο κόμβος είναι το
\lstinline!nextTen[nodeDepth - 1]!.

\item Το πλακίδιο που θα τοποθετηθεί αμέσως μετά από τον άλλο παίκτη είναι το
\lstinline!nextTen[nodeDepth]!.

\item Το πλακίδιο που θα τοποθετηθεί μετά από τον πρώτο παίκτη είναι
\lstinline!nextTen[nodeDepth + 1]!.
\end{itemize}

Άμα ο παίκτης που θα παίξει το \lstinline!nextTen[nodeDepth]! είναι ίδιος με αυτόν που έχει το \lstinline!tile! τότε
τα επόμενα εχθρικά νούμερα στον \lstinline!nextTen! είναι τα:
\begin{center}
\lstinline!nextTen[nodeDepth + 1]!, \lstinline!nextTen[nodeDepth + 3]!, ...
\end{center}
Αλλιώς, είναι τα:
\begin{center}
\lstinline!nextTen[nodeDepth]!, \lstinline!nextTen[nodeDepth + 2]!, ...
\end{center}
Με αυτά γεμίζουμε τη \lstinline!ArrayList<Integer> nextEnemies!.
\begin{lstlisting}[style=chunk]
final boolean tileIsOurs = (tileId == id);
int idxStart = nodeDepth
        + (MinMaxPlayer.nodeLevelIsOurs(nodeDepth) != tileIsOurs ? 1 : 0);

final int[] nextTen = Board.getNextTenNumbersToBePlayed();
final ArrayList<Integer> nextEnemies = new ArrayList<Integer>();
for (int nextTenIdx = idxStart; nextTenIdx < nextTen.length; nextTenIdx += 2) {
    nextEnemies.add(nextTen[nextTenIdx]);
}
\end{lstlisting}

\item Καλούμε την
\hyperref[fun:calculateRisk]{\lstinline!calculateRisk()!}
για να υπολογίσουμε την "ασφάλεια" του \lstinline!tile!.
\begin{lstlisting}[style=chunk]
final double risk = calculateRisk(tile, nodeBoard, nextEnemies);
\end{lstlisting}

\item Προσθέτουμε στο \lstinline!nodeEvaluation! τη διαφορά του score του \lstinline!tile! με το \lstinline!risk!.
Τα πλακίδια του αντιπάλου προστίθενται με αντίθετο πρόσημο από τα δικά μας.
\begin{lstlisting}[style=chunk]
nodeEvaluation += (tile.getScore() - risk) * (tileIsOurs ? 1 : -1);
\end{lstlisting}
\end{itemize}

\subsection{Συνάρτηση \texttt{calculateRisk()}}\label{fun:calculateRisk}
