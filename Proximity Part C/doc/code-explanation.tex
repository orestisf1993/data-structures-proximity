\chapter{Υλοποίηση Των Συναρτήσεων}
\secion{Κλάση \texttt{MinMaxPlayer}}
\subsection{Συνάρτηση \texttt{findEmptyTiles()}}\label{fun:findEmptyTiles}
\begin{lstlisting}[title={Declaration της συνάρτησης}]
private static ArrayList<Tile> findEmptyTiles(final Board board, final int nextDepth)
\end{lstlisting}
Δέχεται το \lstinline!board! και το βάθος \lstinline!nextDepth! του υποδέντρου που δημιουργείται και επιστρέφει ένα
\lstinline!ArrayList<Tile>! με τα ελεύθερα πλακίδια στο \lstinline!board! απορρίπτοντας τα πολλαπλά "μοναχικά" πλακίδια.

Η δομή της συνάρτησης είναι η εξής:
\begin{itemize}
\item Δημιουργούμε μία λίστα για να φορτώνουμε τα πλακίδια που θα επιστρέψουμε.
\begin{lstlisting}[style=chunk]
final int loneTilesLimit = MAX_DEPTH + 1 - nextDepth;
\end{lstlisting}

\item Πραγματοποιούμε μια επανάληψη πάνω σε κάθε πλακίδιο του \lstinline!board!.
\begin{lstlisting}[style=chunk]
for (int i = 0; i < ProximityUtilities.NUMBER_OF_COLUMNS; i++) {
    for (int j = 0; j < ProximityUtilities.NUMBER_OF_ROWS; j++) {
        final Tile tile = board.getTile(i, j);
\end{lstlisting}

\item Για να συμπεριλάβουμε ένα \lstinline!tile! θα πρέπει να είναι κενό
(\lstinline!tile.getPlayerId() == 0!)
και να μην έχουμε ήδη συμπεριλάβει αρκετά "μοναχικά" πλακίδια.
Ένα πλακίδιο το θεωρούμε "μοναχικό" αν όλοι οι γείτονες του είναι κενοί.
Αυτό γίνεται με την συνάρτηση \lstinline!tileIsLone()!.
Στο συνολικό δέντρο θέλουμε να έχουμε \lstinline!MAX_DEPTH! "μοναχικά" πλακίδια έτσι το όριο μέσα στη συνάρτηση είναι:
\lstinline!final int loneTilesLimit = MAX_DEPTH + 1 - nextDepth;!
Με αυτή την επιλογή μειώνουμε σημαντικά τον αριθμό των πλακιδίων για τον οποίο δημιουργούμε το δέντρο μας,
βελτιώνοντας σημαντικά τον χρόνο απόφασης για κάθε κίνηση.
Η ποιότητα της κίνηση δεν μειώνεται γιατί το evaluation μεταξύ των μοναχικών πλακιδίων είναι γενικά ισάξιο.

\item Επιστρέφουμε την λίστα.
\begin{lstlisting}[style=chunk]
return emptyTiles;
\end{lstlisting}
\end{itemize}

\subsection{Συνάρτηση \texttt{createSubTree()}}
\begin{lstlisting}[style=declaration]
void createSubTree(final Node77968125 parent)
\end{lstlisting}
Η συνάρτηση αυτή δημιουργεί ένα υποδέντρο με συνολικό βάθος \lstinline!MAX_DEPTH!.
Δέχεται ως όρισμα τον κόμβο \lstinline!parent! πού είναι ο κόμβος κάτω από τον οποίο αρχίζει το δέντρο.

Η δημιουργία του υποδέντρου γίνεται ως εξής:
\begin{itemize}
\item Βρίσκουμε τα ελεύθερα πλακίδια μέσω της \hyperref[fun:findEmptyTiles]{\lstinline!findEmptyTiles()!}.
Αυτά τα πλακίδια θα γίνουν τα παιδιά του \lstinline!parent!.
\begin{lstlisting}[style=chunk]
final Board board = parent.getNodeBoard();
final int nextDepth = parent.getNodeDepth() + 1;
final int nodeId = nodeLevelIsOurs(nextDepth) ? id : opponentId;
final int s = nextNumbersToBePlayed[nextDepth - 1];
final ArrayList<Tile> emptyTiles = findEmptyTiles(board, nextDepth);
\end{lstlisting}

\item Αρχίζουμε μία επανάληψη για κάθε πλακίδιο στη λίστα
\begin{lstlisting}[style=chunk]
for (final Tile emptyTile : emptyTiles){
\end{lstlisting}

\item Βρίσκουμε το επόμενο \lstinline!board! μέσω της \lstinline!ProximityUtilities.boardAfterMove()!.
\begin{lstlisting}[style=chunk]
final int x = emptyTile.getX();
final int y = emptyTile.getY();
final int[] move = new int[] { x, y };

// Simulate putting a tile on this spot on the parent node board.
final Board nextBoard = ProximityUtilities.boardAfterMove(nodeId, board, x, y, s);
\end{lstlisting}

\item Δημιουργούμε ένα νέο κόμβο στο δέντρο και τον περνάμε σαν παιδί του \lstinline!parent!.
\begin{lstlisting}[style=chunk]
// Create the new node.
final Node77968125 newNode = new Node77968125(parent, nextDepth, move, nextBoard);

// Add the node as child of the parent node.
parent.addChild(newNode);
\end{lstlisting}

\item Άμα δεν έχουμε ξεπεράσει το \lstinline!MAX_DEPTH! καλούμε την \lstinline!createSubTree()! αναδρομικά με νέο πατέρα-κόμβο τον κόμβο που μόλις δημιουργήσαμε.
\begin{lstlisting}[style=chunk]
// Add opponent's branches.
if (nextDepth < MAX_DEPTH) {
    createSubTree(newNode);
}
\end{lstlisting}
\end{itemize}
