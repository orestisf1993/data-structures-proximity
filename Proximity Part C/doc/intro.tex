\chapter{Περιγραφή του προβλήματος}
Στο τρίτο παραδοτέο καλούμαστε να υλοποιήσουμε τον αλγόριθμο MinMax για τη βελτιστοποίηση του παίκτη μας.
Σκοπός είναι να δημιουργήσουμε ένα δέντρο βάθους 2 κινήσεων (τουλάχιστον), το όποιο,
με τη βοήθεια της τροποποιημένης συνάρτηση αξιολόγησης του δεύτερου παραδοτέου θα επιλέγει την καλύτερη δυνατή κίνηση για τον επόμενο γύρο.

Οι συναρτήσεις  που κληθήκαμε να υλοποιήσουμε στην κλάση \lstinline!Node! ήταν οι εξής:
\begin{itemize}
\item Constructors της κλάσης \lstinline!Node!.

\item Getters και τους setters των μεταβλητών της κλάσης \lstinline!Node!.

\item Η συνάρτηση \lstinline!evaluate()! η οποία αξιολογεί την κατάσταση του ταμπλό μετά από την κίνηση.
\end{itemize}

Οι συναρτήσεις  που κληθήκαμε να υλοποιήσουμε στην κλάση \lstinline!MinMaxPlayer! ήταν οι εξής:
\begin{itemize}
\item Η συνάρτηση \lstinline!chooseMinMaxMove()! στην οποία διασχίζουμε το δέντρο που έχουμε υλοποιήσει με
την βοήθεια του αλγορίθμου αναζήτησης κατά βάθος και επιστρέφουμε έναν πίνακα με τις
συντεταγμένες της καλύτερης θέσης σύμφωνα με τον αλγόριθμο minMax.

\item Η συνάρτηση \lstinline!createSubTree()! που δημιουργεί το δέντρο.

\item Η συνάρτηση \lstinline!getNextMove()! η καλείται σε κάθε γύρο και συνδυάζει τις 2 παραπάνω συναρτήσεις και επιστρέφει την επόμενη κίνηση.
\end{itemize}
