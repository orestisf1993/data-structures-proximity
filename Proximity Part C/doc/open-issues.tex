\chapter{Ανοιχτά θέματα}
Διάφορες πιθανές βελτιώσεις του κώδικα είναι:
\begin{itemize}
    \item Βελτιώσεις στην απόδοση του κώδικα.
    Καλύτερες επιδόσεις μας επιτρέπουν να αυξάνουμε το \lstinline!MAX_DEPTH! και έτσι να έχουμε έναν πιο δυνατό παίκτη.
    \begin{itemize}
        \item Η μόνη διαφορά μεταξύ των \lstinline!nodeBoard! σε κόμβους ίδιου επιπέδου είναι το πλακίδιο στη θέση \lstinline!nodeMove! και οι γείτονες του.
        Έτσι θα μπορούσαμε να υπολογίζουμε μια φορά το evaluation για το \lstinline!nodeBoard! του \lstinline!parent! και να αλλάζουμε την αξιολόγηση μόνο στα σημεία που έχουν διαφορές.
        Κάτι παρόμοιο μπορεί να γίνει και με την \lstinline!findEmptyTiles(board, nextDepth)!

        \item A-B pruning.
    \end{itemize}

    \item Μπορούμε να δημιουργήσουμε μια βοηθητική κλάση για να κρατάμε τις \lstinline!static! συναρτήσεις που χρησιμοποιούνται και από τη \lstinline!MinMaxPlayer! και από τη \lstinline!Node77968125!.

    \item Η
    \hyperref[fun:findEmptyTiles]{\lstinline!findEmptyTiles()!}
    θα ήταν καλύτερο να βάζει πρώτα τα ακριανά πλακίδια ώστε να είναι σίγουρη η χρησιμοποίηση τους στο δέντρο.
    Τα ακριανά πλακίδια είναι γενικά καλύτερες επιλογές καθώς έχουν λιγότερους γείτονες και άρα περισσότερη "ασφάλεια".
\end{itemize}
