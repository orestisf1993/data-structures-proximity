\chapter{Υλοποίηση Του Αλγορίθμου}
\begin{sloppypar}
Ο αλγόριθμος αποτελείται από την κλάση \lstinline!HeuristicPlayer! στην οποία υλοποιούνται οι  
συναρτήσεις  \lstinline!getNextMove()!,
\lstinline!getEvaluation()! (οι οποίες ζητούνται από την εκφώνηση) και η 
\lstinline!calculateRisk()! (μία δίκη μας συνάρτηση, βοηθητική στην \lstinline!getEvaluation()!).
Με τις συναρτήσεις αυτές, αρχικά, αξιολογούμε την θέση.
Αυτό γίνεται με την \lstinline!calculateRisk()! και την 
\lstinline!getEvaluation()!.
Αντλώντας πληροφορίες για τους γείτονες, τις διαθέσιμες τιμές που είναι 
πιθανό να έρθουν για την επόμενη κίνηση  για τον αντίπαλο (και για εμάς) και τον αριθμό 
των γειτονικών tiles των γειτόνων μας,
βρίσκουμε έναν αριθμό που αξιολογεί κάθε κενή θέση στο ταμπλό
(δηλαδή κάθε πιθανή επόμενη θέση μας).
Όσο μεγαλύτερος ο αριθμός, τόσο ευνοϊκότερη η θέση.
Στη συνέχεια με την συνάρτηση \lstinline!getNetMove()! βρίσκουμε τη 
θέση με το μεγαλύτερο αριθμό αξιολόγησης και έπειτα τοποθετούμε το στρατό μας σε αυτή.
Επίσης,στην κλάση υπάρχουν, ήδη, υλοποιημένοι οι constructors.
Τέλος, να σημειωθεί ότι δημιουργήσαμε, μία ακόμη επιπλέον, συνάρτηση την \lstinline!updateOpponentsPool()! για να 
παρακολουθούμε τις διαθέσιμες κινήσεις του αντιπάλου.
\end{sloppypar}

\section{\lstinline[basicstyle=\ttfamily]!getNextMove()!}
Η συνάρτηση αυτή δέχεται ως ορίσματα ένα αντικείμενο τύπου \lstinline!Board! και 
την μεταβλητή \lstinline!randomNumber! και επιστρέφει έναν 
μονοδιάστατο πίνακα(\lstinline!result[3]!) με τις συντεταγμένες \lstinline!x,y! 
που αντιστοιχούν στην θέση με την καλύτερη αξιολόγηση καθώς και τον 
\lstinline!randomNumber!.

Η δομή της συνάρτησης είναι η εξής:\begin{itemize}
\item Καλούμε τη συνάρτηση\lstinline! updateOpponentPool()! την 
οποία αναλύουμε στο \hyperref[sec:updateOpponentsPool]{\ref{sec:updateOpponentsPool}}

\item Καλούμε τη συνάρτηση \lstinline!getMyPool()! η οποία επιστρέφει ένα αντικείμενο τύπου \lstinline!HashMap! που αναπαριστά το σύνολο των διαθέσιμων τιμών που είναι πιθανό να έρθουν για την επόμενη κίνηση
\begin{lstlisting}[numbers=none, aboveskip=\smallskipamount, belowskip=\smallskipamount, captionpos=none]
myPool = board.getMyPool();
\end{lstlisting}

\item Αρχή μιας λούπας που προσπελάζει κάθε διαφορετικό πλακίδιο στο ταμπλό
\begin{lstlisting}[numbers=none, aboveskip=\smallskipamount, belowskip=\smallskipamount, captionpos=none]
for (int i = 0; i < ProximityUtilities.NUMBER_OF_ROWS; i++) {
    for (int j = 0; j < ProximityUtilities.NUMBER_OF_COLUMNS; j++) {
\end{lstlisting}
\item Σε κάθε επανάληψη της καλούμε την συνάρτηση \lstinline!board.getTile()! που επιστρέφει ένα αντικείμενο τύπου \lstinline!Tile!. Αυτό είναι το πλακίδιο στη θέση \lstinline!j,i!.
\begin{lstlisting}[numbers=none, aboveskip=\smallskipamount, belowskip=\smallskipamount, captionpos=none]
final Tile tile = board.getTile(j, i);
\end{lstlisting}
\item Αν η θέση είναι κενή (άρα μπορούμε να τοποθετήσουμε το νέο μας πλακίδιο εκεί),
τότε καλούμε την \lstinline!getEvaluation()! και αποθηκεύουμε την επιστρεφόμενη τιμή 
στην μεταβλητή \lstinline!evaluation!.
\begin{lstlisting}[breaklines=true, numbers=none, aboveskip=\smallskipamount, belowskip=\smallskipamount, captionpos=none]
if (tile.getPlayerId() == 0) {
    final double evaluation = getEvaluation(board, randomNumber, tile);
\end{lstlisting}
\item Αν η \lstinline!evaluation! είναι μεγαλύτερη της 
\lstinline!max! (μία μεταβλητή την οποία έχουμε αρχικοποιήσει αρχικά στη τιμή \lstinline!Double.NEGATIVE_INFINITY!) αποθηκεύουμε 
τις συντεταγμένες του πλακιδίου στον πίνακα \lstinline!result!.
\begin{lstlisting}[breaklines=true, numbers=none, aboveskip=\smallskipamount, belowskip=\smallskipamount, captionpos=none]
if (evaluation >= max) {
	max = evaluation;
	result[0] = tile.getX();
	result[1] = tile.getY();
}
\end{lstlisting}
\item Όταν τελειώσει η \lstinline!for! θα έχουμε αποθηκευμένο στη \lstinline!max! το μεγαλύτερο \lstinline!evaluation! και στον πίνακα \lstinline!result! τις 
αντίστοιχες συντεταγμένες του πλακιδίου με το μεγαλύτερο \lstinline!evaluation!.
Τοποθετούμε στη τρίτη θέση του πίνακα result το \lstinline!randomNumber!.
\begin{lstlisting}[breaklines=true, numbers=none, aboveskip=\smallskipamount, belowskip=\smallskipamount, captionpos=none]
result[2] = randomNumber;
\end{lstlisting}
\item Επιστρέφουμε τον πίνακα \lstinline!result!.
\end{itemize}

\section{\lstinline[basicstyle=\ttfamily]!updateOpponentsPool()!} \label{sec:updateOpponentsPool}