\chapter{Υλοποίηση Του Αλγορίθμου}
\section{Συνοπτική Περιγραφή Αλγορίθμου}
\begin{sloppypar}
Ο αλγόριθμος αποτελείται από την κλάση \lstinline!HeuristicPlayer! στην οποία υλοποιούνται οι  
συναρτήσεις  \lstinline!getNextMove()!,
\lstinline!getEvaluation()! (οι οποίες ζητούνται από την εκφώνηση) και η 
\lstinline!calculateRisk()! (μία δίκη μας συνάρτηση, βοηθητική στην \lstinline!getEvaluation(!)).
Με τις συναρτήσεις αυτές, αρχικά, αξιολογούμε την θέση.
Αυτό γίνεται με την \lstinline!calculateRisk()! και την 
\lstinline!getEvaluation()!.
Αντλώντας πληροφορίες για τους γείτονες, τις διαθέσιμες τιμές που είναι 
πιθανό να έρθουν για την επόμενη κίνηση  για τον αντίπαλο (και για εμάς) και τον αριθμό 
των γειτονικών tiles των γειτόνων μας,
βρίσκουμε έναν αριθμό που αξιολογεί κάθε κενή θέση στο ταμπλό
(δηλαδή κάθε πιθανή επόμενη θέση μας).
Όσο μεγαλύτερος ο αριθμός, τόσο ευνοϊκότερη η θέση.
Στη συνέχεια με την συνάρτηση \lstinline!getNetMove()! βρίσκουμε τη 
θέση με το μεγαλύτερο αριθμό αξιολόγησης και έπειτα τοποθετούμε το στρατό μας σε αυτή.
Επίσης,στην κλάση υπάρχουν, ήδη, υλοποιημένοι οι constructors.
Τέλος, να σημειωθεί ότι δημιουργήσαμε, μία ακόμη επιπλέον, συνάρτηση την \lstinline!updateOpponentsPool()! για να 
παρακολουθούμε τις διαθέσιμες κινήσεις του αντιπάλου.
\end{sloppypar}