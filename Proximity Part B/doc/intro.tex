\chapter{Περιγραφή του προβλήματος}
Στη δεύτερη εργασία είχαμε να υλοποιήσουμε μια συνάρτηση αξιολόγησης των διαθέσιμων
κινήσεων που έχει o παίκτης σε κάθε γύρο παιχνιδιού και την αντίστοιχη συνάρτηση 
επιλογής της καλύτερης κίνησης. 
Σε αντίθεση με το πρώτο παραδοτέο, όπου το σκορ για 
κάθε πλακίδιο ήταν εντελώς τυχαίο, πλέον κάθε παίκτης έχει μια δεξαμενή από $60$ πούλια
(3 πούλια για κάθε τιμή από το $1$ έως το $20$).
Πριν από κάθε κίνηση επιλέγεται από τον υπολογιστή ένα πούλι με τυχαία τιμή από όσες είναι διαθέσιμες στην δεξαμενή του παίκτη.
Με βάση αυτή την τιμή εκτελείται η επόμενη κίνηση.
Με την χρήση της δεξαμενής διασφαλίζεται ότι και οι 2 παίκτες θα έχουν παίξει ακριβώς τα ίδια πλακίδια, απλά με διαφορετική (τυχαία) σειρά.
Ουσιαστικά, ο \lstinline!HeuristicPlayer! είναι η εξέλιξη της 
προηγούμενης υλοποίησης μας (\lstinline!RandomPlayer!).
Έτσι,πρέπει να δημιουργήσουμε ένα παίκτη που δεν θα τοποθετεί τον στόλο του (army) τυχαία στις κυψέλες (tiles)
αλλά αξιολογώντας τη θέση θα επιλέγει την βέλτιστη δυνατή θέση (σύμφωνα πάντα,
με τον αλγόριθμο που έχουμε υλοποιήσει).
Δεδομένου αυτού,δεν έχουμε μόνο να αντιμετωπίσουμε την άρτια υλοποίηση 
των βασικών συναρτήσεων αλλά συγχρόνως  να καταστρώσουμε μία καλή στρατηγία.
Για να θεωρηθεί ο αλγόριθμος μας επιτυχημένος, ο παίκτης μας πρέπει να νικάει τον 
\lstinline!RandomPlayer!.
