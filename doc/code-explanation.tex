\chapter{Επίλυση}
Υλοποίησαμε τις εξής μεθόδους της κλάσης RandomPlayer:
\begin{enumerate}
\item (2) Constructors\\
O πρώτος constructor παίρνει ως όρισμα μία μεταβλήτη (\lstinline!pid!) τύπου \lstinline!Integer!
και αρχικοποιεί την μεταβλητή \lstinline!id!.
Ο δεύτερος constructor παίρνει ως
ορίσματα τέσσερις μεταβλητές
(\lstinline!Integer pid!,
\lstinline!String pname!,
\lstinline!int pscore!,
\lstinline!int pnumOfTiles!)
και αρχικοποιεί τις μεταβλήτες \lstinline!id!, \lstinline!name!, \lstinline!score!, \lstinline!numOfTiles!
\item Getters και Setters μεταβλήτων\\
Δημιουργήσαμε τις συναρτήσεις \lstinline!getId()!, \lstinline!getName()!, \lstinline!getNumOfTiles()!,
\lstinline!getScore()! οι οποίες επιστρέφουν τις μεταβλητές \lstinline!id!, \lstinline!name!, \lstinline!score!,
\lstinline!numOfTiles! αντίστοιχα.
Επίσης, υλοποιήσαμε τους setters των μεταβλητών που δίνουν τιμές στις αντίστοιχες μεταβλητές.
\item Συνάρτηση \lstinline!getNextMove()!

\begin{lstlisting}[caption={Η συνάρτηση \lstinline!getNextMove()!}, breaklines=true, numbers=none]
public int[] getNextMove(Board board) {
	boolean tileIsNotEmpty;
	int x, y;
	do {
		x = (int) (Math.random() * ProximityUtilities.NUMBER_OF_COLUMNS);
		y = (int) (Math.random() * ProximityUtilities.NUMBER_OF_ROWS);

		Tile t = board.getTile(x, y);
		tileIsNotEmpty = (t.getPlayerId() != 0);
	} while (tileIsNotEmpty);
	int[] nextMove = { x, y };
	return nextMove;
}
\end{lstlisting}

Με τη βοήθεια της συνάρτησης \lstinline!Math.random()! παίρνουμε δύο τυχαία \lstinline!x!, \lstinline!y!
(στα όρια του ταμπλό) που δίνουν τις συντεταγμένες ενός πλακιδίου.
Στη συνέχεια, με την συνάρτηση \lstinline!getTile()!
της κλάσης \lstinline!Board!
αποθηκεύουμε στη μεταβλητή τύπου \lstinline!Tile!
το αντικείμενο που επιστρέφει η συνάρτηση.
Για να ελέγχουμε αν η θέση είνα κατελειμένη καλούμε την συνάρτηση getPlayerId() της
κλάσης \lstinline!Tile! για το αντικείμενο \lstinline!t! η οποία επιστρέφει $0$ αν το πλακίδιο είναι
κενό.
Συγκεκριμένα ο έλεγχος γίνεται με τη do-while ως εξής:
Ψάχνει νέα x, y μέχρι η \lstinline!getPlayerId()! γίνει $0$ το οποίο σημαίνει ότι βρήκαμε
κενή θέση.
Η συνάρτηση επιστρέφει τον πίνακα με τις συντεταγμένες του πλακιδίου.
\item Συνάρτηση \lstinline!getNeighborsCoordinates()!

\begin{lstlisting}[caption={Η συνάρτηση \lstinline!getNeighborsCoordinates()!}, breaklines=true, numbers=none]
static public int[][] getNeighborsCoordinates(Board board, int x, int y) {
	int[][] neighbors = new int[6][2];
	int[][] adder = (y % 2 == 0) ? addOnEven : addOnOdd;
	for (int i = 0; i < 6; i++) {
		int xn = x + adder[i][0];
		int yn = y + adder[i][1];
		if (board.isInsideBoard(xn, yn)) {
			neighbors[i][0] = xn;
			neighbors[i][1] = yn;
		} else {
			neighbors[i][0] = -1;
			neighbors[i][1] = -1;
		}
	}
	return neighbors;
}
\end{lstlisting}

Η συνάρτηση αυτή παίρνει σαν όρισμα την μεταβλητή \lstinline!board!
και 2 ακεραίους (\lstinline!x!, \lstinline!y!) που αναπαριστούν τις συντεταγμένες ενός πλακιδίου πάνω στο ταμπλό
και επιστρέφει τις συντεταγμένες των γειτονικών πλακιδίων.
Παρατηρούμε ότι οι γείτονες απέχουν από το πλακίδιο σταθερή απόσταση,
ανάλογα αν το πλακίδιο μας βρίσκεται σε άρτια ή περιττή σειρά (\lstinline!y! συντεταγμένη).

Έτσι, αποθηκεύουμε σε ένα πίνακα τις σχέσεις των συντεταγμένων των γειτόνων σε σχέση με τις συντεταγμένες του πλακιδίου.
Ο πίνακας είναι διαμορφωμένος ως εξής:
Στη θέση $(0,0)$ αποθηκεύεται η σταθερή απόσταση του \lstinline!x! του πλακιδίου από το \lstinline!x! του ανατολικού γείτονα
και στη θέση $(0,1)$ η σταθερή απόσταση του \lstinline!y!.
Mε ωρολογιακή φορά σε κάθε στήλη του πίνακα αποθηκεύουμε τις σταθερές αποστάσεις των γειτόνων ξεκινώντας από τον ανατολικό.
Κάθε φορά που καλείται η συνάρτηση με τη βοήθεια των συντεταγμένων \lstinline!x!, \lstinline!y!
ελέγχουμε αν το πλακίδιο μας βρίσκεται σε άρτια ή περιττή σειρά.
Αν βρίσκεται σε άρτια σειρά χρησιμοποιούμε τον πίνακα  \lstinline!addOnEven! ενώ αν βρίσκεται σε περιττή τον πίνακα \lstinline!addOnOdd!.

\begin{lstlisting}[caption={Οι πίνακες \lstinline!addOnEven! και \lstinline!addOnOdd!}, breaklines=true, numbers=none]
	// offsets of the neighboring tiles
	/** The Constant addOnEven used for even y coordinates. */
	final static int[][] addOnEven = { { 1, 0 }, { 0, 1 }, { -1, 1 }, { -1, 0 }, { -1, -1 }, { 0, -1 } };

	/** The Constant addOnOdd used for odd y coordinates. */
	final static int[][] addOnOdd = { { 1, 0 }, { 1, 1 }, { 0, 1 }, { -1, 0 }, { 0, -1 }, { 1, -1 } };
\end{lstlisting}

\begin{sloppypar}
Εν συνεχεία, προσθέτουμε στις συντεταγμένες του πλακιδίου σε κάθε στήλη του πίνακα και καλούμε την συνάρτηση \lstinline!board.isInsideBoard()!.
Αν η \lstinline!board.isInsideBoard()! δώσει τιμή \lstinline!true! είμαστε εντός των ορίων του πίνακα οπότε επιστρέφεται ο πίνακας με τους γείτονες.
Αν δώσει τιμή \lstinline!false! σημαίνει το βρεθήκαμε έξω από τα όρια του ταμπλό οπότε για το συγκεκριμένο γείτονα θα αλλάξουμε τις συντεταγμένες του σε \lstinline![-1,-1]!
Τελικά, επιστρέφεται ο πίνακας των γειτόνων.
\end{sloppypar}
\end{enumerate}
